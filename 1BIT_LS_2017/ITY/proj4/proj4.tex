%%%%%%%%%%%%%%%%%% Alena Tesarova%%%%%%%%%%%%%%%%%%%%%%%%%%%
\documentclass[11pt,a4paper]{article}
\usepackage[utf8]{inputenc}
\usepackage[left=2cm,text={17cm, 24cm},top=3cm]{geometry}
\usepackage{times}
\usepackage[czech]{babel}
\usepackage[T1]{fontenc}

\author{Alena Tesařová}


\begin{document}

\begin{titlepage}

\begin{center}
\Huge
\textsc{Fakulta informačních technologií\\
Vysoké učení technické v~Brně}\\
\vspace{\stretch{0.382}}
\LARGE ITY -- 4. projekt\\
\medskip
{\Huge Bibliografie}
\vspace{\stretch{0.618}}
\end{center}
{\Large \today \hfill
Alena Tesařová}
\end{titlepage}


\section{\LaTeX}

\subsection{Úvodní slovo do krásného jazyka}
\LaTeX je krásný jazyk otvírající bránu do trochu nového světa zpracování textů. Není těžké se s~ním naučit pracovat a je to dobrý pomocník na vytváření matematických vzorců, tabulek, sázení obrázků a skrývá i mnohem víc. Dokonce řada amerických vědců jako například John C. Bowman, Andy Hammerlindl a Tom Prince se zabývají spojováním jazyku C++ s~latexem při tvorbě matematické grafiky. Vytvořili program Asymptote, na kterém už pracují od roku 2004 a od té doby se Asymptote stále vyvíjí. \cite{gstug:matknihovna} \\
Autorem LaTeXu je Donald E.~Knuth ze Standfordské univerzity. \cite{Rybicka:Latex_pro_zacatecniky}


\subsection{Jak psát v~latexu?}
Psaní v~LaTeXu bychom mohli přirovnat k~psaní HTML stránek. Zdrojové texty si \textbf{nemůžeme} ihned prohlédnout jako například v~MS Wordu, ale nejprve je musíme zkompilovat a vytvořit univerzálnější formát, který si budeme moci vytisknout, poslat a dále zpracovávat. \cite{Hordejcuk:Latex} \\
Jako v~klasickém programovacím jazyku může LaTeX načítat soubory s~argumenty a být spuštěný z~příkazového řádku, což je jeho další výhodou. \cite{gstug:spousteni}

\subsection{\textsc{Bib}\LaTeX}
Bibtex je jedna z~možností vytváření citací v~rozsáhlejších dokumentech. Samotný text vytváříme v~souboru .tex, kde se odkazujeme na citace umístěny v~souboru .bib klíčovým slovem \verb|\cite{klic}|.  \cite{Helmut:GuiToLatex} \\
Tato skutečnost vede k~vyšší škálovatelnosti, znovupoužitelnosti, flexibilitě a efektivní manipulaci s~bibiliografickými citacemi. \cite{Luptak:Bibtex} \\
Jak se používá? Předpokladem je správně nainstalovaný program bibtex. Je součástí většiny distribucí LaTeXu, takže by neměl být problém s~jeho zprovozněním. Dále je potřeba mít soubor s~BibTeXovou databází ve stejném adresáři, jako zdrojový text LaTeXového dokumentu \cite{Martinek:Latex}. BibTeXovou databázi je třeba vložit do LaTeXového zdrojáku pomocí příkazu \verb|\bibliography| a upřesnit styl sázení bibliografie příkazem \verb|\bibliographystyle|. Existují obecné styly zabudované v~LaTexu, anebo můžeme najít volně stažitelné styly na internetu například ze stránek Ing.~Martinka. 


\subsection{Slovo na konec}
Důležité je si uvědomit, že LaTex není editor ani nástroj pro sazbu barevných časopisů, ale je to sázecí autorský systém vhodný na sázení bakalářských, diplomových či disertačních prací. \cite{FEKT:Latex} \\
Před psaním práce si stojí za to přečíst práci od Ing. Radka Pyšného viz \cite{Pysny:Bibtex}, který se ve své práci zabýval citacemi a podtrhuje největší chyby při vytváření citací.
Existuje zároveň několik konferencí pořádaných rok co rok zabývajících se rozvojem a využitím tohoto jazyka pořádaných skupinou \textbf{TeX Users Group}. Není užasné, co všechno se dá pomocí tohoto jazyka vytvořit? \cite{Sojka:Hyphenation}


%2 monografie:
%1. rybicka LaTeX D:\Dokumenty\programovani\Internetove-aplikace\LaTeX


\newpage
% styl od R. Pysneho
\bibliographystyle{czplain.bst}
\bibliography{proj4}


\end{document}